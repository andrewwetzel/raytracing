\documentclass{article}
\usepackage{amsmath}
\usepackage{geometry}
\geometry{a4paper, margin=1in}

\title{Key Equations for Ionospheric Ray Tracing}
\author{}
\date{}

\begin{document}

\maketitle

\section{Introduction}

This document provides a list of key equations used in the simulation of radio wave propagation through the Earth's ionosphere.

\section{Refractive Index of the Ionosphere}

The refractive index of the ionosphere, $n$, is given by the Appleton-Hartree equation. A simplified version for a cold, collisionless plasma is:

\begin{equation}
n^2 = 1 - \frac{X}{1 - Y_L \pm \sqrt{Y_T^2 + (Y_L - Y_T)^2}}
\end{equation}

where:
\begin{itemize}
    \item $X = \frac{\omega_p^2}{\omega^2}$
    \item $Y = \frac{\omega_c}{\omega}$
    \item $\omega_p$ is the plasma frequency
    \item $\omega_c$ is the electron gyrofrequency
    \item $\omega$ is the wave frequency
    \item $Y_L = Y \cos\theta$
    \item $Y_T = Y \sin\theta$
    \item $\theta$ is the angle between the wave propagation direction and the Earth's magnetic field
\end{itemize}

\section{Snell's Law}

Snell's Law describes the relationship between the angles of incidence and refraction for a wave passing through a boundary between two different isotropic media.

\begin{equation}
n_1 \sin\theta_1 = n_2 \sin\theta_2
\end{equation}

where:
\begin{itemize}
    \item $n_1$ and $n_2$ are the refractive indices of the two media
    \item $\theta_1$ and $\theta_2$ are the angles of incidence and refraction, respectively
\end{itemize}

\end{document}
